\documentclass[a4paper]{article}

\usepackage[french]{babel}
\usepackage[T1]{fontenc}
\usepackage[utf8]{inputenc}


\begin{document}

	\title{\textbf{Systèmes et réseaux \\ Projet Réseaux de Kahn}}
	\author{Axel Davy \\ \'Ecole normale supérieure \and Baptiste Lefebvre \\ \'Ecole normale supérieure}
	\date{26 mai 2013}
	\maketitle


\section{Choix techniques}
Aucun.

\section{Difficultés rencontrées}
Aucune.

\section{\'Eléments non réalisés}
Aucun.

\section{Installation}

Une fois l'archive \texttt{davy-lefebvre.tgz} récupérée et les fichiers extraits placez-vous dans le répertoire \texttt{davy-lefebvre}. Dans ce répertoire la commande :
\begin{itemize}
	\item \texttt{make} ou \texttt{make thread} : compile notre programme avec l'implémentation naïve qui utilise la bibliothèque de threads d'OCaml
	\item \texttt{make pipe} : compile notre programme avec l'implémentation reposant sur l'utilisation de processus Unix communiquant par des tubes
	\item \texttt{make network} : compile notre programme avec l'implémentation s'exécutant à travers le réseau
	\item \texttt{make sequential} : compile notre programme avec l'implémentation séquentielle où le parallélisme est simulé par notre programme
\end{itemize}
Pour la désinstallation la commande :
\begin{itemize}
	\item \texttt{make clean} : efface tous les fichier qu'une des commandes précédentes a pu engendrer et ne laisse dans le répertoire que les fichiers sources
\end{itemize}
Si vous n'avez pas réussi à récupérer l'archive vous pouvez également récupérer le code source à l'aide de la commande
\begin{itemize}
	\item \texttt{git clone https://github.com/KahnProcessNetworks/KahnProcessNetworks}
\end{itemize}
en vérifiant à bien avoir installé au préalable le système de contrôle de version \texttt{git}.
 


\end{document}

